%!TEX TS-program = xelatex
%!TEX encoding = UTF-8 Unicode


%Template modified by Author:
%Joseph Vargovich 8/4/2020

%Features removed/added
%Objective
%Blue highlighted links for GH, Email, and LinkedIn
%Consistent Section Headers (no half colored headers)
%Conistent bolding of name in header. 
%Other minor formatting changes, such as spacing.
%Quote in header.

%Credit goes to author below, the original author of the template. 


% Awesome CV LaTeX Template for CV/Resume
%
% This template has been downloaded from:
% https://github.com/posquit0/Awesome-CV
%
% Author:
% Claud D. Park <posquit0.bj@gmail.com>
% http://www.posquit0.com
%
% Template license:
% CC BY-SA 4.0 (https://creativecommons.org/licenses/by-sa/4.0/)
%



%-------------------------------------------------------------------------------
% CONFIGURATIONS
%-------------------------------------------------------------------------------
% A4 paper size by default, use 'letterpaper' for US letter
\documentclass[11pt, a4paper]{awesome-cv}

% Configure page margins with geometry
\geometry{left=1.4cm, top=.8cm, right=1.4cm, bottom=1.8cm, footskip=.5cm}

% Specify the location of the included fonts
\fontdir[fonts/]

% Color for highlights
% Awesome Colors: awesome-emerald, awesome-skyblue, awesome-red, awesome-pink, awesome-orange
%                 awesome-nephritis, awesome-concrete, awesome-darknight
%\colorlet{awesome}{awesome-skyblue}
% Uncomment if you would like to specify your own color
\definecolor{awesome}{HTML}{165f99}

% Colors for text
% Uncomment if you would like to specify your own color
% \definecolor{darktext}{HTML}{414141}
% \definecolor{text}{HTML}{333333}
% \definecolor{graytext}{HTML}{5D5D5D}
% \definecolor{lighttext}{HTML}{999999}

% Set false if you don't want to highlight section with awesome color
\setbool{acvSectionColorHighlight}{false}

% If you would like to change the social information separator from a pipe (|) to something else
\renewcommand{\acvHeaderSocialSep}{\quad\textbar\quad}


%Highlighting links in the resume text
 \definecolor{Teal}{RGB}{0,128,128}
 \definecolor{linkblue}{RGB}{0,102,204}
 \usepackage{hyperref}
 \hypersetup{colorlinks,breaklinks,
             urlcolor=linkblue,
             linkcolor=linkblue}

%-------------------------------------------------------------------------------
%	PERSONAL INFORMATION
%	Comment any of the lines below if they are not required
%-------------------------------------------------------------------------------
% Available options: circle|rectangle,edge/noedge,left/right
 %\photo[rectangle,edge,right]{profile}
\name{\textbf{Joseph Vargovich}}{}
\position{Machine Learning Research Assistant{\enskip\cdotp\enskip}Computer Science Student}
%\address{11346 E. Ramblewood Ave. Mesa, Arizona 85212}

\mobile{480-721-1241}
\email{jrv233@nau.edu}
%\homepage{www.posquit0.com}
\github{DovahCraft}
\linkedin{josephvargovich}
% \gitlab{gitlab-id}
% \stackoverflow{SO-id}{SO-name}
% \twitter{@twit}
% \skype{skype-id}
% \reddit{reddit-id}
% \medium{madium-id}
% \googlescholar{googlescholar-id}{name-to-display}
%% \firstname and \lastname will be used
% \googlescholar{googlescholar-id}{}
% \extrainfo{extra informations}


%\quote{``Be disciplined, not motivated."}




%-------------------------------------------------------------------------------
\begin{document}

% Print the header with above personal informations
% Give optional argument to change alignment(C: center, L: left, R: right)
\makecvheader[C]

% Print the footer with 3 arguments(<left>, <center>, <right>)
% Leave any of these blank if they are not needed
\makecvfooter
  {\today}
  {Joseph Vargovich~~~·~~~Resume}
  {\thepage}


%-------------------------------------------------------------------------------
%	CV/RESUME CONTENT
%	Each section is imported separately, open each file in turn to modify content
%-------------------------------------------------------------------------------
%Remove a bit of the space at top if needed
\vskip .05cm 
%-------------------------------------------------------------------------------
%	SECTION TITLE
%-------------------------------------------------------------------------------
\cvsection{Objective}


%-------------------------------------------------------------------------------
%	CONTENT
%-------------------------------------------------------------------------------
\begin{cvparagraph}

%---------------------------------------------------------
Pioneer the development of innovative software solutions that drive a variety of modern research fields.
Create optimized software that is \,innovative and impactful in the information age.

\end{cvparagraph}
\vskip -.2cm

\begin{cvskills}

%---------------------------------------------------------
  \cvskill
    {Core Skills:} % Category
    {Java, C, C++, JavaScript, HTML5/CSS, Python, R, Linux, LaTeX, and Agile methodologies. Proficient in Spanish.} % Skills
\end{cvskills}

\vskip -.2cm
%-------------------------------------------------------------------------------
%	SECTION TITLE
%-------------------------------------------------------------------------------
\cvsection{Education}


%-------------------------------------------------------------------------------
%	CONTENT
%-------------------------------------------------------------------------------
\begin{cventries}

%---------------------------------------------------------
  \cventry
    {B.S. in Computer Science (Honors); \,Masters in CS planned for 2022} % Degree
    {Northern Arizona University} % Institution
    {\textbf{Flagstaff, Arizona}} % Location
    {\textbf{Aug. 2017 - May. 2022}} % Date(s)
    {
      \begin{cvitems} % Description(s) bullet points
        \item {\textbf{GPA:} 3.85 (Dean's List)}
        \item {\textbf{Scholarships/Awards:} LumberJack Scholarship (2017-Present); Dean’s List (2017-Present); Pheatt Family Research and Design Award (2020);
        Nackard Family Honors Scholarship (2020); Perko Family Honors Scholarship (2019); Google Favorite App Award (2015)}
      \end{cvitems}
    }

%---------------------------------------------------------
\end{cventries}

%-------------------------------------------------------------------------------
%	SECTION TITLE
%-------------------------------------------------------------------------------
\cvsection{Work Experience}


%-------------------------------------------------------------------------------
%	CONTENT
%-------------------------------------------------------------------------------
\begin{cventries}

%---------------------------------------------------------
  \cventry
    {Machine Learning Research Assistant} % Job title
    {NAU SICCS Machine Learning Lab} % Organization
    {\textbf{Flagstaff, Arizona}} % Location
    {\textbf{Oct. 2018 - Present}} % Date(s)
    {
      \begin{cvitems} % Description(s) of tasks/responsibilities
        \item {Working within a diverse lab team to develop Machine Learning algorithms for accurate, scalable, and cost-effective genomic data analysis for cancer diagnosis and prediction using C++ and R.}
        \item {Proved time/space complexity of new change point detection algorithms and implemented necessary data structures for efficient computation.}
      \end{cvitems}
    }

%---------------------------------------------------------
  \cventry
    {IT Engineering Intern} % Job title
    {PetSmart} % Organization
    {\textbf{Phoenix, Arizona}} % Location
    {\textbf{May. 2020 - Present}} % Date(s)
    {
      \begin{cvitems} % Description(s) of tasks/responsibilities
        \item {This internship with PetSmart was an IT Engineering internship that entailed shallow learning concepts in Python. However, the official internship was modified due to COVID-19 concerns. The internship transitioned into a remote Summer Experience where I designed IT systems using Agile methodologies such as Scrum.}
        {}
      \end{cvitems}
    }

%---------------------------------------------------------
  \cventry
    {Student Branch Chair} % Job title
    {Northern Arizona University IEEE} % Organization
    {\textbf{Flagstaff, Arizona}} % Location
    {\textbf{Oct. 2018 - Oct. 2019}} % Date(s)
    {
      \begin{cvitems} % Description(s) of tasks/responsibilities
        \item {Led a team of officers to hold Computer Science and Electrical Engineering related workshops, events, and hackathons.}
        \item {Improved event attendance by 40\% through tracking of student interests and demographics.}
        \item {Held workshops on C programming to fill in a knowledge gap required for advanced courses such as Operating Systems.}
      \end{cvitems}
    }


\end{cventries}

%%-------------------------------------------------------------------------------
%	SECTION TITLE
%-------------------------------------------------------------------------------
\cvsection{Honors \& Awards}


%-------------------------------------------------------------------------------
%	SUBSECTION TITLE
%-------------------------------------------------------------------------------
\cvsubsection{International}


%-------------------------------------------------------------------------------
%	CONTENT
%-------------------------------------------------------------------------------
\begin{cvhonors}

%---------------------------------------------------------
  \cvhonor
    {Finalist} % Award
    {DEFCON 26th CTF Hacking Competition World Final} % Event
    {Las Vegas, U.S.A} % Location
    {2018} % Date(s)

%---------------------------------------------------------
  \cvhonor
    {Finalist} % Award
    {DEFCON 25th CTF Hacking Competition World Final} % Event
    {Las Vegas, U.S.A} % Location
    {2017} % Date(s)

%---------------------------------------------------------
  \cvhonor
    {Finalist} % Award
    {DEFCON 22nd CTF Hacking Competition World Final} % Event
    {Las Vegas, U.S.A} % Location
    {2014} % Date(s)

%---------------------------------------------------------
  \cvhonor
    {Finalist} % Award
    {DEFCON 21st CTF Hacking Competition World Final} % Event
    {Las Vegas, U.S.A} % Location
    {2013} % Date(s)

%---------------------------------------------------------
  \cvhonor
    {Finalist} % Award
    {DEFCON 19th CTF Hacking Competition World Final} % Event
    {Las Vegas, U.S.A} % Location
    {2011} % Date(s)

%---------------------------------------------------------
\end{cvhonors}


%-------------------------------------------------------------------------------
%	SUBSECTION TITLE
%-------------------------------------------------------------------------------
\cvsubsection{Domestic}


%-------------------------------------------------------------------------------
%	CONTENT
%-------------------------------------------------------------------------------
\begin{cvhonors}

%---------------------------------------------------------
  \cvhonor
    {3rd Place} % Award
    {WITHCON Hacking Competition Final} % Event
    {Seoul, S.Korea} % Location
    {2015} % Date(s)

%---------------------------------------------------------
  \cvhonor
    {Silver Prize} % Award
    {KISA HDCON Hacking Competition Final} % Event
    {Seoul, S.Korea} % Location
    {2017} % Date(s)

%---------------------------------------------------------
  \cvhonor
    {Silver Prize} % Award
    {KISA HDCON Hacking Competition Final} % Event
    {Seoul, S.Korea} % Location
    {2013} % Date(s)

%---------------------------------------------------------
\end{cvhonors}

%%-------------------------------------------------------------------------------
%	SECTION TITLE
%-------------------------------------------------------------------------------
\cvsection{Presentation}


%-------------------------------------------------------------------------------
%	CONTENT
%-------------------------------------------------------------------------------
\begin{cventries}

%---------------------------------------------------------
  \cventry
    {Presenter for <Hosting Web Application for Free utilizing GitHub, Netlify and CloudFlare>} % Role
    {DevFest Seoul by Google Developer Group Korea} % Event
    {Seoul, S.Korea} % Location
    {Nov. 2017} % Date(s)
    {
      \begin{cvitems} % Description(s)
        \item {Introduced the history of web technology and the JAM stack which is for the modern web application development.}
        \item {Introduced how to freely host the web application with high performance utilizing global CDN services.}
      \end{cvitems}
    }

%---------------------------------------------------------
  \cventry
    {Presenter for <DEFCON 20th : The way to go to Las Vegas>} % Role
    {6th CodeEngn (Reverse Engineering Conference)} % Event
    {Seoul, S.Korea} % Location
    {Jul. 2012} % Date(s)
    {
      \begin{cvitems} % Description(s)
        \item {Introduced CTF(Capture the Flag) hacking competition and advanced techniques and strategy for CTF}
      \end{cvitems}
    }

%---------------------------------------------------------
\end{cventries}

\newcommand{\ts}{\textsuperscript}

%-------------------------------------------------------------------------------
%	SECTION TITLE
%-------------------------------------------------------------------------------
\cvsection{Projects}


%-------------------------------------------------------------------------------
%	CONTENT
%-------------------------------------------------------------------------------
\begin{cventries}

%---------------------------------------------------------
  \cventry
    {Machine Learning Research Publication }% Affiliation/role
    {Linear Time Dynamic Programming Algorithm for the Exact Path of Optimal Models}     
    {\textbf{Mar 2020}}%Organization/group
    {} % Date(s)
    {
      \begin{cvitems} % Description(s) of experience/contributions/knowledge
        \item {Developed and proved a new linear time dynamic programming algorithm for selecting the exact path of optimal changepoint segmentation models from a finite set provided by binary segmentation. \href{https://arxiv.org/abs/2003.02808}{View the publication here.}}
        \item { Observed a \textbf{200\% (x4) speedup} of processing time over previous quadratic time algorithms. Algorithm was developed using C++ and R.}
      \end{cvitems}
    }
    
 \cventry
    {OS development with C} % Affiliation/role
    {Operating System Simulator} % Organization/group
    {\textbf{Apr 2020}}%Location
    {} % Date(s)
    {
      \begin{cvitems} % Description(s) of experience/contributions/knowledge
        \item {Implemented a C based Operating System Simulator that simulates process selection algorithms, multithreading, concurrent processing with context switching, and memory management. Developed exclusively within a Linux command-line environment.}
        \item {Demonstrated clean coding practices by developing a C codebase with 1000+ lines of code with no memory leaks reported through valgrind.}
      \end{cvitems}
    }
    
    \cventry
    {Android Development with Java} % Affiliation/role
    {Chore Tracker Android App} % Organization/group
    {\textbf{Apr 2020}}%Location
    {} % Date(s)
    {
      \begin{cvitems} % Description(s) of experience/contributions/knowledge
        \item {Created an app that incentivises users to complete household chores. \textbf{The app achieved 1\ts{st} place of 6 competing teams} when presented in a mock investor meeting due to its outstanding user interface design and marketable concept. Developed using Java, SQLite, and AWS.}
      \end{cvitems}
     }
 \cventry
    {Arduino Robotics} % Affiliation/role
    {Line Following Robot} % Organization/group
    {\textbf{Mar 2018}}%Location
    {} % Date(s)
    {
      \begin{cvitems} % Description(s) of experience/contributions/knowledge
        \item {Formed a team with four other NAU IEEE members to create a C++ based Arduino robot that followed a preset path to completion.}
        \item{\textbf{Placed 2\ts{nd} of 8 teams} in attendance as the bot completed 90\% of the path.}
      \end{cvitems}
     }
\end{cventries}





%-------------------------------------------------------------------------------
\end{document}
